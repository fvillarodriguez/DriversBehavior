\documentclass[11pt]{article}
\usepackage[utf8]{inputenc}
\usepackage[english]{babel}
\usepackage{amsmath, amssymb}
\usepackage{geometry}
\usepackage{booktabs}
\usepackage{hyperref}
\usepackage{graphicx}
\usepackage{float}

\geometry{a4paper, margin=2.5cm}

\title{Details of Feature Computation for Clustering}
\author{SUMO Team}
\date{\today}

\begin{document}

\maketitle

\section{Introduction}
This document details the *feature engineering* process used to cluster drivers (license plates) based on their behavior on the highway. The goal is to transform raw vehicle flow records into a single feature vector per license plate.

\section{Data Preprocessing}
Before computing the variables, the flow data ($F$) go through a cleaning stage:

\begin{enumerate}
    \item \textbf{License Plate Normalization:} Non-alphanumeric characters are removed and the plate is standardized to uppercase. Invalid plates (e.g., "NULL", "NAN") or plates with length outside the range $[5, 6]$ are discarded.
    \item \textbf{Lane Filtering:} Only lanes 1, 2, and 3 are considered.
    \item \textbf{Duplicate Removal:} Duplicate records are removed considering the tuple $(Plate, Gantry, Lane, Timestamp)$.
    \item \textbf{Outlier Handling (Speed):} \textit{Winsorization} is applied to speed for each $(Gantry, Lane)$ group, capping extreme values at the $1\%$ and $99\%$ percentiles.
\end{enumerate}

\section{Variable Computation (Event Level)}
For each event $i$ (a vehicle passing through a gantry), intermediate metrics are computed. Let $v_i$ be the speed of event $i$ at time $t_i$.

\subsection{Headway ($h_i$)}
\textit{Headway} is computed as the time difference with the preceding vehicle ($i-1$) in the \textbf{same lane and gantry}. The data must be temporally ordered.

\begin{equation}
    h_i = t_i - t_{i-1}
\end{equation}

If $h_i > 60$ seconds, it is assumed there is no close interaction and the value is discarded (NaN) for conflict computation purposes, although it may be included in averages if desired.

\subsection{Indexed Relative Speed ($v_{rel, i}$)}
To normalize behavior according to traffic conditions, the vehicle speed is compared with the average flow speed at that gantry over a 5-minute interval, denoted $\bar{V}_{5min}$.

\begin{equation}
    v_{rel, i} = \frac{v_i}{\bar{V}_{5min}}
\end{equation}

\begin{itemize}
    \item $v_{rel, i} > 1$: Driver is faster than the average flow.
    \item $v_{rel, i} < 1$: Driver is slower than the average flow.
\end{itemize}

\subsection{Conflict and TTC (Time To Collision)}
Collision risk with the preceding vehicle is evaluated. $TTC_i$ is computed only if vehicle $i$ is traveling faster than the previous vehicle ($v_i > v_{i-1}$) and the headway is valid.

\begin{equation}
    TTC_i = \frac{h_i \cdot v_i}{v_i - v_{i-1}}
\end{equation}

A \textbf{conflict} event ($c_i$) is defined if $TTC_i$ is below a gantry-specific threshold ($TTC_{max}$), defined empirically.

\begin{equation}
    c_i = \begin{cases} 
    1 & \text{if } TTC_i < TTC_{max} \\
    0 & \text{if } TTC_i \ge TTC_{max} \text{ or } v_i \le v_{i-1}
    \end{cases}
\end{equation}

\section{Aggregation by License Plate}
The goal is to obtain a single vector per driver $P$. The aggregation process can be performed in two ways: Global or Monthly-Weighted.

Let $N_P$ be the total number of recorded passes for license plate $P$.

\subsection{Aggregated Variables}

\subsubsection{Average Speed ($\bar{v}_P$)}
\begin{equation}
    \bar{v}_P = \frac{1}{N_P} \sum_{i \in P} v_i
\end{equation}

\subsubsection{Average Relative Speed ($\bar{v}_{rel, P}$)}
\begin{equation}
    \bar{v}_{rel, P} = \frac{1}{N_{rel}} \sum_{i \in P} v_{rel, i}
\end{equation}
Where $N_{rel}$ is the number of events with valid relative speed.

\subsubsection{Average Headway ($\bar{h}_P$)}
\begin{equation}
    \bar{h}_P = \frac{1}{N_{h}} \sum_{i \in P, h_i \le 60} h_i
\end{equation}

\subsubsection{Conflict Rate ($CR_P$)}
Proportion of events with valid headway that resulted in a conflict.
\begin{equation}
    CR_P = \frac{\sum_{i \in P} c_i}{N_{h}}
\end{equation}

\subsubsection{Lane Usage ($Lane_k$)}
Proportion of usage of lane $k \in \{1, 2, 3\}$.
\begin{equation}
    Lane_k = \frac{\text{Count of passes in lane } k}{N_P}
\end{equation}

\subsubsection{Lane Change Rate ($LCR_P$)}
Estimated by counting how many times the vehicle changed lanes between consecutive gantries. Let $L_j$ be the lane in event $j$ (time-ordered).
\begin{equation}
    Changes = \sum_{j=2}^{N_P} \mathbb{I}(L_j \neq L_{j-1})
\end{equation}
\begin{equation}
    LCR_P = \frac{Changes}{N_P - 1}    
\end{equation}
*Note: This metric is approximate, as it assumes continuity between recorded gantries.*

\subsection{Consolidation Strategies}

\subsubsection{1. Global Aggregation (Default)}
The formulas above are applied over \textbf{all} historical records for license plate $P$ as a single set.

\subsubsection{2. Monthly Weighting}
Monthly averages $X_{P, m}$ are computed first for each month $m$, and then weighted by the number of trips in that month ($N_{P, m}$).

Let $X$ be any variable (e.g., Speed, Conflict Rate). The final weighted value is:

\begin{equation}
    X_{P, final} = \frac{\sum_{m} \left( X_{P, m} \cdot N_{P, m} \right)}{\sum_{m} N_{P, m}}
\end{equation}

This ensures that months with higher activity have greater influence, while still allowing intermediate monthly metrics if temporal analysis is required.

\section{Definition of Frequent Drivers}
For clustering model training, users are separated into:
\begin{itemize}
    \item \textbf{Frequent:} Users with enough history to characterize their behavior. Typical criteria:
    \begin{itemize}
        \item $N_P \ge 20$ total passes.
        \item Active days $\ge 5$.
    \end{itemize}
    \item \textbf{Infrequent:} Users with few data points. They are not used for training, but can be assigned to a cluster later using the trained model (or remain as "Unknown").
\end{itemize}

\section{Python Implementation}
The computation logic described above is implemented mainly in the module \texttt{src/clustering.py}. Below, we detail the execution flow of the key functions for auditing purposes.

\subsection{Data Cleaning: \texttt{clean\_flujos\_for\_clustering}}
This function takes the raw flow DataFrame and performs the initial preparation:
\begin{enumerate}
    \item \textbf{Column Validation:} Ensures that essential columns exist (timestamp, speed, gantry, lane, license plate).
    \item \textbf{License Plate Cleaning:} Plates are normalized (alphanumeric, uppercase) and filtered if their length is invalid (outside the range $[5, 6]$).
    \item \textbf{Basic Filtering:} Rows with null values in critical fields are removed and only lanes 1, 2, and 3 are kept.
    \item \textbf{Deduplication:} Exact duplicates of the tuple (Plate, Gantry, Lane, Timestamp) are removed. If duplicates exist with different speeds, a deterministic priority is applied (merge sort ordering preserving the first).
    \item \textbf{Winsorization:} Extreme speed values are capped by group (Gantry, Lane) using the defined percentiles (default $1\%$ and $99\%$).
\end{enumerate}

\subsection{Feature Computation: \texttt{Clusterization}}
This is the main function that orchestrates feature computation.

\subsubsection{1. Simple Metrics and Pre-computation}
Metrics that do not require complex temporal ordering are computed:
\begin{itemize}
    \item \textbf{Total Passes (\texttt{total\_passes}):} Simple count of records per license plate.
    \item \textbf{Average Speed (\texttt{avg\_speed\_kmh}):} Arithmetic mean of speed.
    \item \textbf{Lane Usage:} A pivot table is built to count passes per lane and divide by the total.
\end{itemize}

\subsubsection{2. Relative Speed}
\begin{itemize}
    \item The average flow speed is computed in 5-minute windows per gantry (\texttt{interval\_speed\_mean}).
    \item This average is joined to each event to compute \texttt{relative\_speed} $= v_i / \bar{V}_{5min}$.
\end{itemize}

\subsubsection{3. Interaction Metrics (Headway, Conflict)}
This stage is critical and sensitive to data ordering:
\begin{enumerate}
    \item \textbf{Sorting:} Data are strictly sorted by Gantry, Lane, and Timestamp.
    \item \textbf{Vectorized Operations (Shift):} Shifted columns (\texttt{shift(1)}) are created to compare the current event with the immediately preceding vehicle in the same lane/gantry.
    \item \textbf{Headway Computation:} Time difference between the current and previous event. Headways greater than \texttt{max\_headway\_s} (60s) are discarded (NaN).
    \item \textbf{Conflict Computation:}
        \begin{itemize}
            \item Compute $TTC = \frac{Headway \cdot v_{current}}{v_{current} - v_{previous}}$.
            \item Valid only if $v_{current} > v_{previous}$.
            \item Mark conflict if $TTC < TTC_{max}$ (where $TTC_{max}$ depends on the gantry).
        \end{itemize} 
    \item \textbf{Aggregation:} Valid events are summed and counted by license plate to obtain averages.
\end{enumerate}

\subsubsection{4. Lane Changes (\texttt{Lane Changes})}
\begin{itemize}
    \item Data are sorted by License Plate and Timestamp.
    \item The current lane is compared with the previous one. If they differ (and it is the same plate), it is counted as a change.
    \item The rate is normalized by ($N_P - 1$).
\end{itemize}

\subsubsection{5. Weighting (If applicable)}
If \texttt{monthly\_weighting} is enabled, metrics are computed first by grouping by Plate-Month, and then a weighted average is taken using the number of trips in the month to obtain the final value for the plate.

\end{document}