\documentclass[11pt]{article}
\usepackage[utf8]{inputenc}
\usepackage[spanish]{babel}
\usepackage{amsmath, amssymb}
\usepackage{geometry}
\usepackage{booktabs}
\usepackage{hyperref}
\usepackage{graphicx}
\usepackage{float}

\geometry{a4paper, margin=2.5cm}

\title{Detalle del Cálculo de Variables para Clustering}
\author{Equipo SUMO}
\date{\today}

\begin{document}

\maketitle

\section{Introducción}
Este documento detalla el proceso de ingeniería de características (*feature engineering*) utilizado para agrupar conductores (patentes) en función de su comportamiento en la autopista. El objetivo es transformar los registros brutos de flujos vehiculares en un vector de características único por patente.

\section{Preprocesamiento de Datos}
Antes de calcular las variables, los datos de flujo ($F$) pasan por una etapa de limpieza:

\begin{enumerate}
    \item \textbf{Normalización de Patentes:} Se eliminan caracteres no alfanuméricos y se estandariza a mayúsculas. Patentes inválidas (ej. "NULL", "NAN") o con longitud fuera del rango $[5, 6]$ son descartadas.
    \item \textbf{Filtrado de Carriles:} Se consideran únicamente los carriles 1, 2 y 3.
    \item \textbf{Eliminación de Duplicados:} Se eliminan registros duplicados considerando la tupla $(Patente, Portico, Carril, Timestamp)$.
    \item \textbf{Tratamiento de Outliers (Velocidad):} Se aplica \textit{Winsorization} a la velocidad por cada grupo de $(Portico, Carril)$, limitando los valores extremos a los percentiles $1\%$ y $99\%$.
\end{enumerate}

\section{Cálculo de Variables (Nivel Evento)}
Para cada evento $i$ (un paso de un vehículo por un pórtico), se calculan métricas intermedias. Sea $v_i$ la velocidad del evento $i$ en el instante $t_i$.

\subsection{Headway ($h_i$)}
El \textit{headway} se calcula como la diferencia de tiempo con el vehículo precedente ($i-1$) en el \textbf{mismo carril y pórtico}. Se requiere que los datos estén ordenados temporalmente.

\begin{equation}
    h_i = t_i - t_{i-1}
\end{equation}

Si $h_i > 60$ segundos, se considera que no hay interacción cercana y el valor se descarta (NaN) para efectos de cálculo de conflicto, aunque puede contabilizarse en promedios si se desea.

\subsection{Velocidad Relativa Indexada ($v_{rel, i}$)}
Para normalizar el comportamiento según las condiciones del tráfico, se compara la velocidad del vehículo con la velocidad promedio del flujo en ese pórtico durante un intervalo de 5 minutos, denotada como $\bar{V}_{5min}$.

\begin{equation}
    v_{rel, i} = \frac{v_i}{\bar{V}_{5min}}
\end{equation}

\begin{itemize}
    \item $v_{rel, i} > 1$: Conductor más rápido que el flujo promedio.
    \item $v_{rel, i} < 1$: Conductor más lento que el flujo promedio.
\end{itemize}

\subsection{Conflicto y TTC (Time To Collision)}
Se evalúa el riesgo de colisión con el vehículo precedente. El $TTC_i$ se calcula solo si el vehículo $i$ viaja más rápido que el vehículo anterior ($v_i > v_{i-1}$) y el headway es válido.

\begin{equation}
    TTC_i = \frac{h_i \cdot v_i}{v_i - v_{i-1}}
\end{equation}

Se define un evento de \textbf{conflicto} ($c_i$) si el $TTC_i$ es menor a un umbral específico por pórtico ($TTC_{max}$), definido empíricamente.

\begin{equation}
    c_i = \begin{cases} 
    1 & \text{si } TTC_i < TTC_{max} \\
    0 & \text{si } TTC_i \ge TTC_{max} \text{ o } v_i \le v_{i-1}
    \end{cases}
\end{equation}

\section{Agregación por Patante}
El objetivo es obtener un vector único por conductor $P$. El proceso de agregación puede realizarse de dos formas: Global o Ponderada Mensual.

Sea $N_P$ el número total de pasadas registradas para la patente $P$.

\subsection{Variables Agregadas}

\subsubsection{Velocidad Promedio ($\bar{v}_P$)}
\begin{equation}
    \bar{v}_P = \frac{1}{N_P} \sum_{i \in P} v_i
\end{equation}

\subsubsection{Velocidad Relativa Promedio ($\bar{v}_{rel, P}$)}
\begin{equation}
    \bar{v}_{rel, P} = \frac{1}{N_{rel}} \sum_{i \in P} v_{rel, i}
\end{equation}
Donde $N_{rel}$ es el número de eventos con velocidad relativa válida.

\subsubsection{Headway Promedio ($\bar{h}_P$)}
\begin{equation}
    \bar{h}_P = \frac{1}{N_{h}} \sum_{i \in P, h_i \le 60} h_i
\end{equation}

\subsubsection{Tasa de Conflicto ($CR_P$)}
Proporción de eventos con headway válido que resultaron en conflicto.
\begin{equation}
    CR_P = \frac{\sum_{i \in P} c_i}{N_{h}}
\end{equation}

\subsubsection{Uso de Carriles ($Lane_k$)}
Proporción de uso del carril $k \in \{1, 2, 3\}$.
\begin{equation}
    Lane_k = \frac{\text{Conteo de pasadas en carril } k}{N_P}
\end{equation}

\subsubsection{Tasa de Cambio de Pista ($LCR_P$)}
Se estima contando cuántas veces el vehículo cambió de carril entre pórticos consecutivos. Sea $L_j$ el carril en el evento $j$ (ordenado por tiempo).
\begin{equation}
    Cambios = \sum_{j=2}^{N_P} \mathbb{I}(L_j \neq L_{j-1})
\end{equation}
\begin{equation}
    LCR_P = \frac{Cambios}{N_P - 1}    
\end{equation}
*Nota: Esta métrica es aproximada ya que asume continuidad entre pórticos registrados.*

\subsection{Estrategias de Consolidación}

\subsubsection{1. Agregación Global (Default)}
Se aplican las fórmulas anteriores sobre \textbf{todos} los registros históricos de la patente $P$ como un solo conjunto.

\subsubsection{2. Ponderación Mensual}
Se calculan primero los promedios $X_{P, m}$ para cada mes $m$, y luego se ponderan por la cantidad de viajes en ese mes ($N_{P, m}$).

Sea $X$ una variable cualquiera (ej. Velocidad, Conflict Rate). El valor final ponderado es:

\begin{equation}
    X_{P, final} = \frac{\sum_{m} \left( X_{P, m} \cdot N_{P, m} \right)}{\sum_{m} N_{P, m}}
\end{equation}

Esto asegura que los meses con mayor actividad tengan mayor influencia, pero permite calcular métricas mensuales intermedias si se requiere análisis temporal.

\section{Definición de Conductores Frecuentes}
Para el entrenamiento del modelo de clustering, se separan los usuarios en:
\begin{itemize}
    \item \textbf{Frecuentes:} Usuarios con suficiente historia para caracterizar su comportamiento. Criterios típicos:
    \begin{itemize}
        \item $N_P \ge 20$ pasadas totales.
        \item Días activos $\ge 5$.
    \end{itemize}
    \item \textbf{Infrecuentes:} Usuarios con pocos datos. No se usan para entrenar, pero pueden ser asignados a un cluster posteriormente usando el modelo entrenado (o quedar como "Desconocido").
\end{itemize}

\end{document}
